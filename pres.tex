\documentclass{beamer}
\setbeamertemplate{caption}[numbered]
\usetheme{Boadilla}
% \usepackage{beamerthemesplit} // Activate for custom appearance

\usepackage{booktabs}
\usepackage{caption}
\usepackage{threeparttable}

\title{Matthew Cocci's Potent Productivity}
\author{Micah Smith}
\date{\today}

\begin{document}

\frame{\titlepage}

\section[Outline]{}
\frame{\tableofcontents}

\section{Introduction}
\subsection{Historical productivity}
\frame
{
  \frametitle{Cocci is very productive}

  \begin{itemize}
  \item<1-> This is pretty obvious to most of us.
  \item<2-> How does he do it?
  \item<3-> No one really knows.
  \end{itemize}
}

\subsection{Motivation of empirical study}
\frame
{
  \frametitle{Motivation of empirical study}

  \begin{itemize}
  \item<1-> Collect data to document productivity for future generations of RAs.
  \item<2-> Conduct inference on causes (and \textit{effects}) of productivity.
    \begin{itemize}
    \item<3-> Cocci is a big fan of the podcast Serial.
    \item<4-> Presumably, his productivity decreased on days in which he was listening to a new Serial episode.
    \item<5-> If we could analyze his productivity over time, we could determine on which days new episodes of Serial were released.
    \end{itemize}
  \item<6-> Embarrass him in front of all his loved ones.

  \end{itemize}
}

\subsection{Description of data}
\frame
{
  \frametitle{Description of data}

  \begin{itemize}
  \item<1-> Cocci uses the \texttt{tmux} utility on the RAN to manage large collections of "panes", or pseudo terminal sessions.
  \item<2-> Make observations of the number of panes that Cocci has open at a given point to get a sense of productivity over time.
  \item<3-> I collect a novel dataset of hourly observations of the number of panes open:
    \begin{itemize}
    \item<4-> Timestamped at 9am through 5pm, seven days a week.
    \item<5-> Data collected from August 11, 2014 to yesterday (July 21, 2015).
    \item<6-> 3100 hourly observations over 345 days in total.
    \item<7-> Don't even worry about how I did this.
    \end{itemize}
  \end{itemize}
}

\section{Pretty tables/pictures}
\subsection{Summary statistics}
\frame
{
  \frametitle{Summary statistics}
  
  \begin{table}
  \caption{Description of hourly data}
  \begin{tabular}{lrrr}
\toprule
{} &   panes &  absd1panes &  sqd1panes \\
\midrule
count & 2181.00 &     2180.00 &    2180.00 \\
mean  &   23.44 &        1.06 &       5.52 \\
std   &   12.66 &        2.09 &      30.54 \\
min   &    0.00 &        0.00 &       0.00 \\
25\%   &   15.00 &        0.00 &       0.00 \\
50\%   &   21.00 &        0.00 &       0.00 \\
75\%   &   30.00 &        1.00 &       1.00 \\
max   &   73.00 &       26.00 &     676.00 \\
\bottomrule
\end{tabular}

    \begin{tablenotes}
      \footnotesize
      \item Hours excluded from sample are non-workdays or days on which the RAN was reset.
    \end{tablenotes} 
  \end{table}
}

\frame
{
  \frametitle{Summary statistics}
  
  \begin{table}[htbp]
  \caption{Description of daily (collapsed) data}
  \begin{tabular}{llrr}
\toprule
{} & has\_episode\_today &  absd1panes &  sqd1panes \\
\midrule
count &               243 &      243.00 &     243.00 \\
mean  &              0.04 &        9.55 &      49.49 \\
std   &              0.20 &        7.94 &      95.27 \\
min   &             False &        0.00 &       0.00 \\
25\%   &              0.00 &        4.00 &       6.00 \\
50\%   &              0.00 &        8.00 &      18.00 \\
75\%   &              0.00 &       13.00 &      54.00 \\
max   &              True &       46.00 &     756.00 \\
\bottomrule
\end{tabular}

      \begin{tablenotes}
      \footnotesize
  \item Days excluded from sample are non-workdays or days on which the RAN was reset. \texttt{has\_episode\_today} is marked if an episode of Serial was released on that day.
    \end{tablenotes} 
    \end{table}
}

\subsection{Graphical trends}
\frame
{
  \frametitle{Hourly trend}
  
  \begin{center}
  \includegraphics[scale=0.35]{Results/Plots/mean_abs_prod_hrly_excl_reset_only_workdays.eps}
  \end{center}
}

\frame
{
  \frametitle{Weekly trend}
  
  \begin{center}
  \includegraphics[scale=0.35]{Results/Plots/mean_abs_prod_excl_reset.eps}
  \end{center}
}

\section{Model}
\frame
{
\frametitle{Model: Description}
\begin{itemize}
\item<1-> Use absolute change in panes as a rough proxy for hourly productivity, and aggregated absolute hourly change in panes as a rougher proxy for daily productivity.
\item<2-> Include squared term to capture all of the super duper non-linear effects we expect.
\item<3-> Control for day of the week as well as Brent Crude spot prices.
\item<3-> Train a boosted decision trees classifier, because why not, and because Cocci deserves only the best.
\item<4-> Train on data from August 11, 2014 to March 31, 2015; test on data from April 1 to yesterday.
\end{itemize}
}

\frame
{
\frametitle{Model: Results}
\begin{table}[htbp]
\caption{Days with potential Serial episode releases}
\begin{tabular}{lr}
\toprule
{} &  pred \\
\midrule
date       &       \\
2015-04-02 &  0.54 \\
2015-05-12 &  0.90 \\
2015-06-18 &  0.62 \\
2015-07-02 &  0.54 \\
2015-07-16 &  0.54 \\
\bottomrule
\end{tabular}

\begin{tablenotes}
\begin{footnotesize}
\item Last \textit{known} episode of Serial released on December 18, 2014.
\end{footnotesize}
\end{tablenotes}
\end{table}

}

\section{Conclusions}
\frame
{
\frametitle{Conclusions}
\begin{itemize}
\item<1-> Based on Cocci's productivity trends, an episode of Serial was released with 90\% certainty on May 12. It's just that no one but him has noticed yet.
\item<2-> Good luck at Princeton!
\item<3-> Do Something Good Everyday!
\end{itemize}
}

\end{document}
